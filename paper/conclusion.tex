\section{Conclusion}

There is quite a bit of attention in the area of program repair and
synthesis right now. Many of the current approaches take a test suite based
approach to understanding what the intended behavior of the program is.
Though we borrow conceptually from some of these techniques, our approach is
fundamentally different in that we rely on program specifications, namely
preconditions and postconditions, to understand the intended behavior of a
program.

Our program repair technique, Jager, relies on symbolic execution and SMT
solvers to repair programs. This is accomplished with 5 distinct steps: 1)
symbolic trace generation, 2) trace differentiation, 3) statistical fault
localization, 4) precondtion and postcondition generation around the fault,
and 5) fine-grained synthesis to replace the fault. Though our
implementation is in its infancy, we believe it provides a sufficient proof
of concept for the approach we have presented in this paper.

As the supporting state of the arts for technologies like symbolic
execution, SMT solvers, and synthesis continue to progress and evolve, we
believe that program repair techniques like ours will become increasingly
relevant. In a world where even the simplest applications can contain large
amounts of code, it is increasingly necessary to have ways of automatically
identifying and repairing bugs so that valuable human resources can be
focused elsewhere.
