\documentclass[]{article}
\usepackage[T1]{fontenc}
\usepackage{lmodern}
\usepackage{amssymb,amsmath}
\usepackage{ifxetex,ifluatex}
\usepackage{fixltx2e} % provides \textsubscript
% use microtype if available
\IfFileExists{microtype.sty}{\usepackage{microtype}}{}
% use upquote if available, for straight quotes in verbatim environments
\IfFileExists{upquote.sty}{\usepackage{upquote}}{}
\ifnum 0\ifxetex 1\fi\ifluatex 1\fi=0 % if pdftex
  \usepackage[utf8]{inputenc}
\else % if luatex or xelatex
  \usepackage{fontspec}
  \ifxetex
    \usepackage{xltxtra,xunicode}
  \fi
  \defaultfontfeatures{Mapping=tex-text,Scale=MatchLowercase}
  \newcommand{\euro}{€}
\fi
\ifxetex
  \usepackage[setpagesize=false, % page size defined by xetex
              unicode=false, % unicode breaks when used with xetex
              xetex]{hyperref}
\else
  \usepackage[unicode=true]{hyperref}
\fi
\hypersetup{breaklinks=true,
            bookmarks=true,
            pdfauthor={},
            pdftitle={},
            colorlinks=true,
            urlcolor=blue,
            linkcolor=magenta,
            pdfborder={0 0 0}}
\urlstyle{same}  % don't use monospace font for urls
\setlength{\parindent}{0pt}
\setlength{\parskip}{6pt plus 2pt minus 1pt}
\setlength{\emergencystretch}{3em}  % prevent overfull lines
\setcounter{secnumdepth}{0}

\author{Joshua Branchaud and Eric Rizzi \\
Department of Computer Science and Engineering \\
University of Nebraska-Lincoln \\
\{jbrancha,erizzi\}@cse.unl.edu}
\date{}

\begin{document}

\title{Jager: Symbolic Specification-Based Program Adjustment}

\maketitle

\section{Introduction}
The field of program synthesis has seen dramatic improvements over the last
few years.  The increasing power of SAT solvers has allowed researchers to
ask more sophisticated questions and to handle larger programs than they
ever have before.  In short, this is an exciting time in program synthesis.
With the increased power available, the field has progressed beyond the
merely theoretical stage, to one that is making real contributions with even
larger gains expected in the future.

For our project, we decided to implement some of the techniques described in
the state of the art research in order to gain a better understanding of
their power and limitations, and to investigate ways to increase the speed
of these methods.  The infrastructure we were eventually able to set up
combined five separate steps to achieve the overall synthesis.  These steps
are symbolic trace gathering, trace separation, fault localization, post bug
symbolic execution, and synthesis.  By combining these five steps, we were
able to achieve a start to finish program repair that ran almost completely
automatically.  While the programs that we were eventually able to repair
were admittedly very small, it was exciting to see even obvious errors
disappear without any oversight.

% We need to think about how we are going to provide a background for the
% work, but right now an explicit background section isn't fitting. It may
% work its way in later though.
%\subsection{Background}

\section{Limitations}
With all of the improvements seen in the past few years, however, the
synthesis community must contend with the basic limitations that the
technique will likely always have to address.  These can be split into two
main classes.  The first set of limitations has to do with the ability to
verify a program as correct or not.  Foremost among these are the difficulty
of writing accurate specifications, the unwillingness of programmers to
write these specifications, and the difficulty, in environments lacking
specifications, of determining what is the intended outcome.  These
fundamental limitations must be at the forefront of the mind of everyone who
works in this field.  They are also limiting factors to how widely these
techniques can be adopted since any critical program is unlikely to use
synthesized programs if the results they produce cannot be veified.

In addition to these verification concerns, there are certain computational
restrictions with which researchers must contend.  The term program
synthesis is a bit of a misnomer, due to the fact that we are not, in fact,
synthesizing an entire program.  To synthesize an entire piece of software
from scratch is at this point impossible for all but the smallest toy
programs.  The reason for this is the difference between verification and
creation.  Verify a program ignores the huge amount of effort that was
required to even get the program close to working.  The current techniques
which attempt to synthesize an entire program take an iterative approach;
creating a program from pieces and checking to see whether this particular
combination was meaningful in any way.  The sheer number of combinations
with which these simplistic pieces can be combined is overwhelming.  The
best techniques use a genetic algorithm approach to build on the successes
of past attempts, but even with this approach, all but the simplest programs
are intractable.  Therefore, instead of synthesizing an entire program, our
technique instead relies on an underlying framework, produced by a
programmer, which is assumed to be mostly correct.  This assumption is based
on the Competent Programmer Hypothesis, and without it, program synthesis
would be too expensive an undertaking to attempt.  From this framework,
however, incremental improvements can be made until the entire program is
proven correct.

\section{Approach}
Before continuing on to an explanation of our technique, there were several
assumptions and restrictions that we made in our project in order to
decrease the complexity of the programs with which we were dealing.  One of
the largest restrictions was the fact that we only tested programs that
didn't contain any type of looping structure.  These loops would have been
problematic for the following reasons.  First, since we generated symbolic
traces via symbolic execution, we would have encountered the same problem
all symbolic execution techniques do.  When encountering loops that are
either unbounded or require a large number of iterations to traverse, we
would have to unroll these loops to a certain finite depth.  This would
result in an approximation of the overall program's behavior, resulting in
an approximate fix.  This lack of certainly about the value of a fix would
be unacceptable for safety critical programs and is something that must be
dealt with before program synthesis can ever be applied in an unsupervised
setting.  Secondly, the loops would cause a problem due to the recursive
properties that would be required of a fix.  Not only would the fix be
required to fix a series of traces, the fix would be required to not
interfere with itself for a variable number of times in order to solve the
unrolled loops.  Consider for example, a loop with an error that is unrolled
2 times.  A fix would have to fix the first trace.  It would also have to be
able to fix the second trace by combining with itself in a constructive way.
This singular fix takes away the concept, seen in several papers, of
incremental improvement.  This would make it almost impossible to fix
multiple bugs that occur within a single loop, as well as more difficult to
solve a single bug.

In addition to only using programs without loops, we decided to limit the
types of bugs we could fix.  We assumed that we would only have to replace
the right hand side of an erroneous assignment.  This meant that the
variable that is being written to is assumed to be correct; merely the value
that is calculated for it to store is incorrect.  This draws upon the
Competent Programmer Hypothesis, allowing us to use the framework the
programmer provided to make progress when complete synthesis would be
computationally intractable.  We assumed that that the types of holes we
would have to fix were either constant values or single variable linear
equations.  While it would have been possible to calculate more difficult
forms of calculations that could occur on the right hand side with the
information that we collect, we didn't implement the grammars required by
the solvers to answer these questions.

Finally, we assumed that there would only be a single buggy line in each
program.  This assumption is due to another weakness of synthesis; the
necessity to make incremental progress.  Synthesis cannot currently change
multiple lines at once.  Instead it must fix one line, see if it works, and
then continue on to the others.  While this approach means that it is would
be possible to fix multiple independent buggy lines of code, fixing
dependent bugs would be much more difficult.  The reason that independent
bugs can be fixed was explored in the work of Logozzo et al. who classify
fixing of a program as the reduction of the number of bad traces.  If the
buggy lines in the program, therefore, were independent, the first fix would
be retained, since it improved the program, allowing full verification upon
fixing the second bug.  If instead, the bug is the result of multiple lines,
no iterative single line change would be able to fix the problem, and
therefore the bug would go unfixed.  We assumed there would be only a single
buggy line in order to avoid the necessity of having to distinguish between
independent and dependent errors.

\subsection{Trace Generation}
The first step in synthesizing a program is generating traces that describe
the program's behavior.  Some of the techniques we read about used concrete
traces, generated from test suites.  Since we assumed that the program would
be accompanied by a comprehensive set of specs, we thought we would achieve
a much more complete understanding of the behavior of the program with
symbolic traces.  Therefore, we created a listener that would output the
complete set of instructions encountered along each symbolic path.  These
paths can be though of as being representative of an entire set of concrete
traces.  Therefore the amount of information that is captured with these
traces is much larger, resulting in a more precise analysis; one that would
be provably correct at the end.  Each of these traces is captured in SSA
form, which makes manipulating them to put into the SAT solver much easier.

\subsection{Trace Differentiation}
In this step of our program, we add pre and post conditions to the traces
that were generated in the previous step and calculate which traces followed
a path of execution resulting in satisfying the post condition, and which do
not.  This is a two-step process.  First, since we didn't included the
preconditions in our listener, we must remove any traces that wouldn't be
possible to follow.  We do this by querying the solver PRE AND TRACE.  This
query is akin to asking the solver, is it possible to start with values that
satisfy the precondition and execute every single instruction in the trace
without encountering any contradictions.  For example, if we generated the
trace (x1<0, x2=x1+1) from our original exploration, but the precondition
was x>0, then this trace would be completely disregarded from consideration
since its behavior is undefined.  For the traces that satisfy the
pre-conditions, there is still one more test to do.  We now must see if it
is possible to start out in the pre-conditions, execute every statement in a
trace, and end up in a space that is not considered acceptable, ie. end up
in a state that is not described by the post condition.  To do this, we ask
the solver an existential question; whether it is possible to end up in an
area that is not the post condition.  For all those traces where this is not
possible (the solver returns UNSAT), we know that these traces are safe and
all possible values that satisfy their constraints will result in satisfying
our post conditions.  For those traces where the solver returns SAT, this
means that exists at least one value that can end up outside the
post-conditions.  Therefore the entire trace is marked as problematic.  The
traces which pass the initial filter are then passed to the next stage of
the synthesis process.  Those where it is possible to end up in a space that
is not the post condition being marked \emph{bad}, the other being marked
\emph{good}.

\subsection{Fault Localization}

As discussed earlier in the limitations, it is important to isolate a very
small portion of the program that can be blamed for the fault. This is where
fault localization comes into the picture. In our case, we are interested in
identifying a single statement that is potentially faulty. It is that faulty
statement that we attempt to adjust in such a way that \emph{fixes} the
program. Like SemFix \cite{}, we borrow from the approach of Jones et al.
\cite{JonesTarantula} by computing a ranked list of \emph{suspicious}
statements. Unlike Jones et al. and SemFix though, we do not use traces from
the runs of test cases in a test suite. Instead, we utilize the traces
produced by symbolic execution.

The approach for localizing faults is to build a ranked list of the most
supsicious statements in the program. We present the suspiciousness formula
used for this kind of fault localization in Equation~\ref{tarantula1}.

\begin{equation}
\label{tarantula1}
    susp(s) = \frac{failing(s)/totalFailing}{failing(s)/totalFailing + passing(s)/totalPassing}
\end{equation}

Equation~\ref{tarantula1} relies entirely on the presence of a test suite
that can be run to differentiate between passing and failing test cases
which refer to concrete traces. It also relies on the test suite being of a
certain quality with the suite having as high of a coverage metric as
possible. Unlike these approaches, our approach uses \emph{good} and
\emph{bad} symbolic traces (discussed in the previous section) to replace
the passing and failing concrete traces. Thus, we present a subtle variant
of the suspiciousness formula with Equation~\ref{tarantula2}.

\begin{equation}
\label{tarantula2}
    susp(s) = \frac{bad(s)/totalBad}{bad(s)/totalBad + good(s)/totalGood}
\end{equation}

By using Equation~\ref{tarantula2} with all the traces through the program, we can
quickly compute a list of statements ranked by suspiciousness. Instead of
being tied to a test suite, the suspiciousness is based on all traces
through the program and their ability to meet (or not) the specifications of
the program. We believe this provides a fuller, more precise picture of what
statements may be culpable. However, we do recognize that just as test
suites can have low coverage, the specifications for the program may not be
strong enough to say much.

Once we have a ranked list of suspicious statements, we can grab the most
suspicious statement and pass it to the next phases of our technique so that
an adjustment can be made. If there are multiple statements with the same
suspiciousness score, we arbitrarily pick one. If a statement doesn't match
the kind of statements we attempt to repair, then we grab the next statement
that does fall in line. This is necessary as long as we have a limited
repair grammar. Additionally, if a suitable repair can not be found for the
first statement passed along, then we pass along a subsequent statement.


\subsection{Pre and Post Condition Generation}
Once the bug has been found, we now go through the process of isolating the
traces that are associated with the fault.  This is a two-part process.
First, we slice all traces, both good and bad, that go through the buggy
line so that anything at or below the bug no longer exists.  The reason for
this is that we must account for all possible ways to approach an error
within the program, but also do not want the information generated after the
bug to effect our synthesis.  Following this, we again symbolically execute
the program, this time, however, only for the paths emanating from the bug.
The reason for this is due to the fact that since the bug caused an
incorrect execution, it is possible that it forced a particular trace to go
down the wrong path following the bug.  It is also possible that a path that
was considered infeasible when executing the buggy line, now may be
possible.  Therefore, we want to consider all possible flows of execution
emanating from the error.  With the completion of this step, we have a set
of top traces and a set of bottom traces that surround the error.  This
information is then passed to the final stage of our process, synthesis, in
order to repair the program.

\subsection{Synthesis}
This final part of our process is by far the most intensive, requiring not
only the creation of a line that will fix the program, but also verification
that this fix works for all paths that are involved with the repair.  The
program pairs each top trace with every possible bottom trace with a series
of series of OR statements.  The result look something like [(T1 AND1 HOLE
AND1 B1) OR (T1 AND1 HOLE AND1 B2) OR ( )...] AND  [(T2 AND1 HOLE AND1 B1)
OR (T2 AND1 HOLE AND1 B2) OR ( )...] AND ... The reason for this is we know
that it is possible to traverse the top traces T1, T2, ..., since we are
assuming that there are no errors within this part of the program.  We also
know they must flow through at least one of the bottom traces (Thus the
boolean AND1's in the inner queries).  What we do not know, however, is
which bottom trace they will eventually progress down.  This is the reason
that we combine the prospective traces with OR's, since only one needs to
hold true in order for the program to work.  Finally, we combine all of the
information about how the program could progress though each of these
individual top traces together with AND2's.  This is because we program must
be able to properly execute along all possible paths that lead to the hole.

The HOLE part of the program is where the synthesis magic comes in.  We are
asking the solver two different questions at this stage.  First, we are
asking whether it is possible for some statement to bridge the gap between
the top and the bottom.  Second, we ask for a model of this statement that
will serve as our prospective repair.  If the answer to the first query is
false, then we know that our grammar is not powerful enough to fill in the
hole.  Another way of saying this is that no statement of the form the
grammar is capable of expressing can fix the program.  If the answer,
however, is true, then it is possible that a statement can bridge the gap.
We ask the solver for a model of this statement, which can be used to
construct a patch.

Unfortunately, our work is not yet done.  What this model represents is an
existential repair.  It basically says that it is possible to start in the
precondition, execute some path, and end up in an acceptable post-condition;
that some value will satisfy all of the given constraints.  Just like our
trace separation stage, however, this assertion is not powerful enough.
Instead, we want to know whether this patch will work for all values of
precondition.  In order to show this, we must negate the post condition, and
see if it is possible to end up outside of the post condition.  If the
solver returns UNSAT to this query, meaning that it is not possible to end
up with a value that doesn't satisfy the post condition, then our work is
done and we have both created and verified the repair.  If instead the
solver returns SAT, this means that it is still possible to end up in an
incorrect final state.  Should this happen, we must return to our original
synthesis, add the information that the repair previously generated doesn't
work, and try again.  Failure to find a repair that works for all possible
values of the precondition right away is not uncommon.  Therefore, we must
iterate between finding a model and seeing if that model works.  This
iterative process can continue either until we run out of resources, or
until we find a repair that works for all traces.

It should be noted that the first query of this stage prevents us from
executing this iterative process on a grammar that couldn't possibly satisfy
all of the necessary constraints.  It basically tells us whether a
particular grammar is powerful enough to fill in the hole.  We just can't
learn which particular form of our grammar will solve it without iterating
through models until we finally find one that works.

%\section{Evaluation}

%\section{Related Work}

\section{Future Work}

Reduce size of queries via...

Removing redundant top traces

Looking for merge points between traces to reduce number of variables used in whole query

Remove impossible bottom traces based on top traces

Basically if variable unassociated with bug is constraint in top and bottom
in infeasible way, then throw it out.

Look into removing variables that aren't involved in bug from consideration

Perhaps through postcondition raising

Integrate steps 1, 2 into single step

Increase complexity of grammars

Investigate dependent bugs by pairing statements

\section{Conclusion}

\bibliographystyle{plain}
\bibliography{bib/paper}

\end{document}
